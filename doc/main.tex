\documentclass{article}
\usepackage{fullpage}

%load needed packages
\usepackage{graphicx}
\usepackage{array}
\usepackage{booktabs}
\usepackage[utf8]{inputenc}
\usepackage[spanish]{babel} % Paquete para el idioma español

\begin{document}



% Portada
\begin{titlepage}
	\centering
	\vspace*{3cm}
	
	% Título destacado
	{\Huge \textbf{Practica 1: Bases De Datos Relacionales}\\[0.5cm]}
	
	% Espacio y logotipo (si lo tienes, por ejemplo el logo de tu universidad)
	\vspace{2cm}
	\includegraphics[width=0.3\textwidth]{images/uma_logo.jpg}\\[1cm]
	
	% Nombre del autor
	{\LARGE \textbf{Alejandro Silva Rodríguez}\\[0.5cm]}
	{\LARGE \textbf{Marta Cuevas Rodríguez}\\[0.5cm]}
	{\large \textit{Almacenes De Datos}\\
		Universidad de Málaga\\
		}
	
	\vfill
	
	% Fecha en la parte inferior de la página
	{\large Septiembre 2024}
\end{titlepage}

% indice
\tableofcontents

\newpage

\section{Introducción}

En esta práctica, se diseñará y creará una base de datos para gestionar información sobre parques naturales en Andalucía y la liga española de fútbol de primera división. El objetivo es elaborar un modelo entidad-relación y un modelo relacional que muestren las características y relaciones de las entidades involucradas, utilizando herramientas de diseño de bases de datos.

Además, se generará el DDL (Data Definition Language) para SQL Server, que permitirá crear la base de datos y las tablas necesarias. Se incluirán datos sobre los parques naturales, su gestión y las especies que los habitan, así como información sobre los equipos de fútbol, sus jugadores y los partidos.

Este ejercicio ayudará a aplicar los conceptos de bases de datos y ofrecerá una experiencia práctica en el uso de herramientas y lenguajes de consulta.



\section{Parques Naturales}

This is the first paragraph. Lorem ipsum dolor sit amet, consectetur adipiscing elit, sed do eiusmod tempor incididunt ut labore et dolore magna aliqua. Ut enim ad minim veniam, quis nostrud exercitation ullamco laboris nisi ut aliquip ex ea commodo consequat. Duis aute irure dolor in reprehenderit in voluptate velit esse cillum dolore eu fugiat nulla pariatur. Excepteur sint occaecat cupidatat non proident, sunt in culpa qui officia deserunt mollit anim id est laborum.

And this is the second paragraph. Lorem ipsum dolor sit amet, consectetur adipiscing elit, sed do eiusmod tempor incididunt ut labore et dolore magna aliqua. Ut enim ad minim veniam, quis nostrud exercitation ullamco laboris nisi ut aliquip ex ea commodo consequat. Duis aute irure dolor in reprehenderit in voluptate velit esse cillum dolore eu fugiat nulla pariatur. Excepteur sint occaecat cupidatat non proident, sunt in culpa qui officia deserunt mollit anim id est laborum.



\subsection{Name Subsection 2.a}

Example of itemize list:


\begin{itemize}
  \item Itemize item 1
  \item Itemize item 2
\end{itemize}



\subsection{Name Subsection 2.2}

Example of enumerate list:

\begin{enumerate}
  \item Enumerate item 1
  \item Enumerate item 2
  \item Enumerate item 3
\end{enumerate}



\section{Liga De Futbol}

Lorem ipsum dolor sit amet, consectetur adipiscing elit, sed do eiusmod tempor incididunt ut labore et dolore magna aliqua. Ut enim ad minim veniam, quis nostrud exercitation ullamco laboris nisi ut aliquip ex ea commodo consequat. Duis aute irure dolor in reprehenderit in voluptate velit esse cillum dolore eu fugiat nulla pariatur. Excepteur sint occaecat cupidatat non proident, sunt in culpa qui officia deserunt mollit anim id est laborum.

\begin{figure}
	\centering
	\includegraphics[width=0.30\textwidth]{images/uma_logo.jpg}
	\caption{Caption of the image}
	\label{fig:example1}
\end{figure}

Reference Figure \ref{fig:example1}.

\cite{oracle2024}
\cite{sqlserver2022}

% Incluir la bibliografía
\bibliographystyle{plain}  % Estilo de la bibliografía (por ejemplo, plain, alpha, ieee, etc.)
\bibliography{biblio}  % Nombre del archivo .bib sin la extensión

\end{document}
